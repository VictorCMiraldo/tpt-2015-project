\documentclass{llncs}
\usepackage[utf8x]{inputenc}
\usepackage[T1]{fontenc}

\usepackage{url}
\usepackage{float}
\usepackage[usenames,dvipsnames]{color}
\usepackage{enumerate}
\usepackage{amsmath}
%\usepackage{amsthm}
\usepackage{amsfonts}
\usepackage{amssymb}
\usepackage{makeidx}
\usepackage{parskip}
\usepackage{multirow}
\usepackage{moreverb}
\usepackage{catchfilebetweentags}
\usepackage{latex/agda}
\usepackage{bussproofs}

\newcommand{\D}[1]{\AgdaDatatype{#1}}
\newcommand{\F}[1]{\AgdaFunction{#1}}
\newcommand{\K}[1]{\AgdaKeyword{#1}}
\newcommand{\N}[1]{\AgdaSymbol{#1}}
\newcommand{\IC}[1]{\AgdaInductiveConstructor{#1}}
\newcommand{\ICArgs}[2]{\AgdaInductiveConstructor{#1}$\; #2 $}
\newcommand{\DArgs}[2]{\D{#1}$\; #2 $}

\DeclareUnicodeCharacter {10627}{\{\hspace {-.2em}[}
\DeclareUnicodeCharacter {10628}{]\hspace {-.2em}\}}
\DeclareUnicodeCharacter {8759}{::}
\DeclareUnicodeCharacter {8718}{$\square$}
\DeclareUnicodeCharacter {8799}{$\stackrel{\tiny ? \vspace{-1mm}}{=}$}
\DeclareUnicodeCharacter {8339}{$_x$}
\DeclareUnicodeCharacter {8336}{$_a$}
\DeclareUnicodeCharacter {7523}{$_r$}
\DeclareUnicodeCharacter {7506}{$^\circ$}
\DeclareUnicodeCharacter {8348}{$_t$}
\DeclareUnicodeCharacter {7522}{$_i$}
\DeclareUnicodeCharacter {9733}{$\star$}
\DeclareUnicodeCharacter {119924}{$\mathcal{M}$}
\DeclareUnicodeCharacter {8872}{$\vDash$}
\DeclareUnicodeCharacter {9656}{$\blacktriangleright$}

\title{Separation in Agda \\ \small{Theory of Programming and Types}}
\author{Victor Cacciari Miraldo}
\institute{University of Utrecht}
\date{\today}

\begin{document}
\maketitle

\newcommand{\Agda}[1]{\ExecuteMetaData[latex/Excerpts.tex]{#1}}
\newcommand{\PMap}[1]{\ExecuteMetaData[latex/PMapParts.tex]{#1}}

% \TODO: \citeSwierstra command!

\newcommand{\citeSwierstra}{\cite{Swierstra2014} }
\newcommand{\citeHoare}{\cite{Hoare69} }
\newcommand{\Htriple}[3]{\{ #1 \} \; #2 \; \{ #3 \}}
\newcommand{\fileid}{F_{id}}
\newcommand{\BN}{\mathbb{N}}
\newcommand{\MIM}{\mathcal{M}}
%\newcommand{\iff}{\Leftrightarrow}


\newcommand{\textSigma}{$\Sigma$}

\section{Introduction}

This document discusses the technicalities of proving \emph{separation} properties in Agda.
Although it is very intuitive to be able to separate things from one another, we have faced
subtle problems that led us to the development of domain specific data structures.

Hoare Logic \citeHoare is a deduction calculus driven by a programming language structure,
it allows one to reason on the lines of \emph{If we run the program $c$ at any state $s$
satisfying $P\;s$, then the resulting state $s'$ is going to satisfy $Q \; s'$}. This is denoted
by $\Htriple{P}{c}{Q}$. This calculus by itself already allows one to prove interesting
results about a program. However, when concurrency gets into play, or whenever we need
to explicitly divide our state in smaller parts (separation), we cannot express these properties anymore.

Separation Logic, on the other hand, introduces a new type of operator, called \emph{separating conjunction}, which
we denote by $\cdot \star \cdot$. For illustration purposes, we will define a toy language for 
managing a folder with text files and a predicate language over it. 

The project proposal mentioned an exploration of \emph{Concurrent Separation Logic} (CSL)
soundness \cite{Vafeiadis2011} in Agda. CSL is an extension of \emph{Separation Logic} \cite{Hearn2001},
targeted at handling the reasoning of concurrent reads and writes to a mutable state. As this
was an experimental project, the end result is somewhat divergent from the original proposal.
CSL soundness turned out to be too complicated to be proved in Agda, leading us to explore
different aspects of \emph{Separation Logic}, mainly targeted at reasoning over \emph{version control systems},
as in \citeSwierstra. 

\section{Deliverable and Report Structure}

The main deliverable that arose from this project is under 
the \emph{data structure with guarantees} category. We developed a library
for handling lists with no duplicate elements, which as later used to
develop a library for handling \emph{separable} partial maps. 
We used this partial map library to model a version control semantics similar
to the one given in section 6 of \citeSwierstra.

The full code is available on GitHub (\url{https://github.com/VictorCMiraldo/tpt-2015-project}).
This repository also contains the some experimentations with CSL. The files which this report regards
are:
\begin{enumerate}[i)]
  \item \texttt{\small Repo.Models.Abs2}, containing the repository model.
  \item \texttt{\small Repo.Data}, where the modules \F{List1} and \F{PMap1} are.
\end{enumerate}

We start by briefly describing some technical details of both libraries, in section \ref{pmaps}. The discussion is then focused on the version control model, section \ref{repo}.
A discussion on some design options is presented on section \ref{discussion}, which
is then followed by our conclusions, in section \ref{conclusion}.

\section{Partial Maps in Agda}
\label{pmaps}

Partial maps are an important ingredient for almost any discrete mathematical definition.
Giving semantics to separation logic, concurrent or not, is not an exception. Therefore,
we need to encode partial maps in Agda. A first option would be to use lists of pairs.
This will fail, however, whenever we try to split a map in two disjoint parts, as Agda
will tell us there might be duplicate elements.

We created a library for handling lists with no duplicates, and therefore partial maps
with no duplicates. Allowing us to divide such maps whenever we see fit. 
There are a few ways to define such notion in Agda, but an inductive definition
is by far the easier to handle, as this will drive the induction of almost every proof.

\PMap{noDups-mod-def}

It is worth mentioning that the \F{noDups-mod} predicate models \emph{no duplicates modulo predicate}. 
In the partial map case, a \F{PMap1} is defined by:

\PMap{pmap1-def}

where $A$ is a module parameter for \texttt{\small Repo.Data.PMap1}, with decidable equality $eqA$. 
Looking something in a map is pretty straight forward, but of paramount importance.

\PMap{lkup-lift-def}

The lifting of a partial map to a \F{Maybe} valued function is very useful. Since
this function representation does not take order into account, we can define
an equality relation modulo a permutation. This will be very useful for
proving commutativity of union, for instance.

\PMap{eq-modulo}

It is trivial to prove that this is an equivalence relation, as transitivity and
symmetry comes directly from $\_\equiv\_$. In order to provide a minimal working library,
in the lines of Haskell's \texttt{\small Data.Map}, we need the following mutually
recursive lemmas.

\PMap{lemmas}

\subsection{Defining Union}

All this code was used mainly to able to model the \emph{set theoretical} 
notion of partition, where a set $S$ can always be divided into two disjoint
subsets $s_1$ and $s_2$ such that $s_1 \cup s_2 \equiv S$. In separation logic, 
however, most of the times one defines the union of maps to be defined if and only if
the parts are disjoint. We have a few different ways of expressing this disjointedness
notion, the most handy one is by induction, again.

\PMap{disjoint-def}

It is useful to note that this notion is proof-irrelevant. Unfortunately we noticed
this fact a bit late in the development of the library. This could greatly simplify
a few proofs. This re-engineering is left as future work.

\PMap{disjoint-pi-def}

This allows us to give union notion and a (provable) way to separate
a map in two disjoint parts, such that we can reconstruct it later with the
previously mentioned union.

\PMap{union-def}

\PMap{focus-star-def}

Without delving in too much implementation details, the library we developed
consists in providing standard operations over \F{List1} and \F{PMap1} that
preserve the \F{noDups-mod} and \F{disjoint} predicate. The code is available
on GitHub, under \url{https://github.com/VictorCMiraldo/tpt-2015-project}.

\section{Modeling a Folder}
\label{repo}

We modified the repository model given in \citeSwierstra, section 6. In order to provide more control
over which \emph{parts} of a file were changed, we define a line to be a pair $\fileid \times \BN$,
where $\fileid$ is an abstract file identifier (we assume we have decidable equality over $\fileid$)
and the second projection represents the line in that file. 

\Agda{file-defs}

A repository, then, is:

\Agda{repo-def}
  
where \F{FileMap.to1} $A$ denotes a partial map $\fileid \rightharpoonup A$.
We chose to model lines as natural numbers. A line $n$ is followed by a line $m$
if and only if $m = \F{suc }n$. In fact, we denote the first component of the
product by \F{files}, and the second as \F{contents}.

The predicate language we are going to use is minimal. It is easy to add and manipulate 
propositional constructs using pre-condition strengthening and post-condition weakening.
An important note, however, is that we did not model the $Empty$ predicate, but opted
to use the \emph{repository does not have the file $f$}. The technicalities of this will be
discussed later. Intuitively, though, it makes sense to be able to add a file to a folder,
as long as there is no naming conflict. This design has the drawback to require,
explicitly in the precondition, that the repository does not contain any of the files we
are going to create. 

\Agda{logic-def}

Note that the \IC{Has} constructor receives two arguments instead of one, as in the
original definition. The predicate (\IC{Has} f $n$) states not only the repository has a file, 
but also that this file has \emph{at least} $n$ lines.

\subsection{Giving it Semantics}

The next step is to formalize which repositories satisfy which formulas. 
Given a repository $m = (m_f , m_c) : \MIM$ and a formula $p : Logic$, we denote
by $m \vDash p$ whenever $m$ satisfies $p$. Case $p$ is,

\begin{align*}
  (m_f , m_c) \vDash & \; (\IC{Hasnt}\; f) \\
               \iff  & f \notin m_f \\[2mm]
  (m_f , m_c) \vDash & \; (\IC{Has}\; f\; n) \\
               \iff  & lkup\;f\;m_f \equiv just\; n' \wedge n \leq n'\\[2mm]
  (m_f , m_c) \vDash & \; ((f , n)\; \IC{is}\; b) \\
               \iff  & m_c (f , n) \equiv just\; b \\[2mm]
  (m_f , m_c) \vDash & \; (p \star q) \\
               \iff  & \exists k_1 , k_2\;\cdot\; k_1 \cup k_2 \approx m_f
                       \wedge k_1 \cap k_2 \equiv \emptyset \\
                     & \wedge (k_1 , m_c) \vDash p
                       \wedge (k_2 , m_c) \vDash q
\end{align*}

The first three lines are intuitive. The last one introduces the first subtlety in the
encoding of separation in Agda. When we state that $k_1 \cup k_2 \approx m_f$, we
assume that order is not important. In Agda, however, propositional equality
of lists takes ordering into account. This is where the equality modulo permutation
fixes our problem.

One interesting result we were able to prove, is that adding \emph{irrelevant}
information to the model does not change the validity of a formula. We call this
the \emph{augmentation} lemma.

\Agda{augmentation-lemma}

This lemma allows us to justify a $\star$-elimination rule, that is,
$p \star q \Rightarrow p$ and $p \star q \Rightarrow q$. The proof is
simple classically, once we have the augmentation lemma. 
Let sketch the proof of the first elimination rule for illustration
purposes: assuming a model $(m_f , m_c) \vDash p \star q$, then
there exists a partition of $m_f$, let's call it $k_1$ and $k_2$.
We know that $(k_1 , m_c) \vDash p$ and we need to prove that
$(k_1 \cup k_2 , m_c) \vDash p$. The fact that $k_1 \cap k_2 = \emptyset$
and the augmentation lemma concludes the proof.

Note that if we provided the $Empty$ predicate in our assertion language, we
would not be able to prove this lemma for all formulas, if $(m_f , m_c) \vDash Empty$,
then $dom\; m_f \equiv []$, and we would need to prove that $dom\;(m_f \cup k) \equiv []$, which
is clearly absurd.

We have a few pros and cons of using this design. It is very handy to be able
to \emph{garbage collect} terms in formulas, making proof-trees much more readable by
a human. The downside being the need to begin the proof asserting that the repository
does not have the files we seed to add in the given patch.

A noteworthy remark is that our repository model is close to the model for Concurrent
Separation Logic given by Vafeiadis in \cite{Vafeiadis2011}, that is, the product
of a partial map and a function. The separation-related properties of CSL
only separate the partial map of the model, which is why we also don't require any change to
the \F{contents} of the repository in the augmentation lemma. 

\subsection{Derivations}

After providing a formal semantics for repositories and assertions over them, we should
introduce our language for managing a repository and state how should we reason about it,
in a derivational style.

\Agda{repo-language-def}

The commands are pretty straight-forward. In figure \ref{calculus} we give the proof rules for our language. 
The soundness proof is left as future work, as it should hold, at least intuitively.
As a simple illustration, let us consider the following patch:

\Agda{repo-1}

If we apply this patch to a repository that has no file named $0$ or $1$, the end result
will contain both files, empty. The proof type is presented below but the proof is omitted
here.

\Agda{proof-repo-1}

\newcommand{\prooflbl}[1]{\RightLabel{\scriptsize{(#1)}}}

\begin{figure}[h]
\begin{minipage}[t]{0.55\textwidth}
  \begin{prooftree}
    \AxiomC{}
    \prooflbl{touch}
    \UnaryInfC{\Htriple{$Hasnt\;f$}{touch f}{$Has\;f\;0$}}
  \end{prooftree} \vspace{2mm}
  \begin{prooftree}
    \AxiomC{}
    \prooflbl{rm}
    \UnaryInfC{\Htriple{$Has\;f\;0$}{rmfile f}{$Hasnt\;f$}}
  \end{prooftree}
\end{minipage}
\begin{minipage}[t]{0.45\textwidth}
  \begin{prooftree}
    \AxiomC{$\Htriple{P}{c}{Q}$}
    \AxiomC{$isFrame\;S\;c$}
    \prooflbl{frame}
    \BinaryInfC{$\Htriple{P * S}{c}{Q * S}$}
  \end{prooftree} \vspace{2mm}
  \begin{prooftree}
    \AxiomC{\Htriple{$P$}{c}{$S$}}
    \AxiomC{\Htriple{$S$}{d}{$Q$}}
    \prooflbl{seq}
    \BinaryInfC{\Htriple{$P$}{$c \blacktriangleright d$}{$Q$}}
  \end{prooftree}
\end{minipage}
\vspace{2mm}
\begin{prooftree}
    \AxiomC{}
    \prooflbl{ins}
    \UnaryInfC{\Htriple{$Has\;f\;n$}{insert (f , n) bs}
                       {$Has\;f\;(suc\;n) \wedge (f, n)\;is\;bs$}}
  \end{prooftree}
  \vspace{2mm}
\begin{prooftree}
    \AxiomC{}
    \prooflbl{rep}
    \UnaryInfC{\Htriple{$Has\;f\;n \wedge (f , n)\;is\;bs$}
                       {replace (f , n) bs bs'}
                       {$Has\;f\;n \wedge (f, n)\;is\;bs'$}}
\end{prooftree}
\vspace{2mm}
\begin{prooftree}
    \AxiomC{$P \Rightarrow P'$}
    \AxiomC{\Htriple{$P'$}{c}{$Q'$}}
    \AxiomC{$Q \Rightarrow Q'$}
    \prooflbl{refine}
    \TrinaryInfC{\Htriple{$P$}{c}{$Q$}}
  \end{prooftree}
\caption{Proof rules of the repository language}
\label{calculus}
\end{figure}

\section{Design Options}
\label{discussion}

We could have opted for a different design in multiple parts of this project. One
important remark is why we did not use \F{Setoid} from the standard library to model
our partial map's equality. The reason for that is the convenience of substitutivity.
If we have propositionally equal lifts, we can \K{rewrite} between them.

Another design detail can be seen in the side condition for the frame rule, in figure \ref{calculus}.
We specify a predicate \F{isFrame}, which, given a formula and a comment returns the \F{Set}
stating that the given formula addresses do not overlap the addresses referenced in the command.
This is easier then stating $addr\;R \cap mod\;c \equiv \emptyset$, as proving inductively defined
properties is easier, and this also allows for a more refined control of what
overlapping addresses actually means. The \F{isFrame} predicate is defined below. 

\Agda{addr-def}
\Agda{isFrame-def}

\section{Conclusions}
\label{conclusion}

During this project we identified the possible pitfalls one might fall for
when trying to prove separation properties in Agda. Although very intuitive,
we could see how this intuition fails us when we move to a constructive world.

The libraries for handling partial maps and lists with no duplicate elements
are easier to use than we thought they would be, once we fixed the equality question.
Their implementation, however, is a bit tricky. The full code is about 1 thousand lines
of Agda. It is true that a proper re-engineering of properties could be made, as we believe
that there are multiple proofs that can be written as implications of previously proved
lemmas.

Unfortunately, we had no time to deeply investigate how our changes to the repository
model affect the overall problems described in \citeSwierstra. Nor could we finish
handling branching and merging. This is left as future work.

The Agda formalization of our repository model makes extensive use of our libraries,
and still have a few (inexplicable) \emph{unsolved metas}. Specially in the proof of the
augmentation lemma. Yet, a pen and paper proof shows us that the lemma really holds,
now it is a matter of explaining it to Agda. 


\bibliographystyle{alpha}
\bibliography{references} 

\end{document}
